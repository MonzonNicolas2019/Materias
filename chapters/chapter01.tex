\section{Ejercicios}

\begin{enumerate}

\item Calcular, siempre que exista, la integral doble $I(f,\mathcal{R}) = \iint_{\mathcal{R}}f(x,y)dxdy$ en el recinto $\mathcal{R} \subset \mathcal{D} \subset \mathbb{R}^2$ del campo escalar $f: \mathcal{D} \rightarrow \mathbb{R}$. Graficar el recinto $\mathcal{R}$ en el que se integra y calcular su \'area $A(\mathcal{R})$. Determinar, adem\'as, el valor medio $$\mu(f,\mathcal{R}) = \frac{\iint_{\mathcal{R}}f(x,y)dxdy}{\iint_{\mathcal{R}}dxdy}$$ del campo escalar en el recinto\footnote{Si, por ejemplo, la funci\'on $f$ representa una densidad superficial, el valor medio en la l\'amina $\mathcal{R}$ es la densidad media. Una l\'amina $\mathcal{R}$ homog\'enea de densidad igual a la densidad media, tiene la misma masa que la l\'amina original de densidad variable dada por la funci\'on $f$}.
\begin{enumerate}
	\item $\mathcal{R} = \left[ 0, 1 \right] \times \left[ 0, 2 \right], f: \mathbb{R}^2 \rightarrow \mathbb{R} \textrm{ tal que } f(x,y) = 3x^2 + 2y$.
\end{enumerate}

\end{enumerate}

\section{Claves de correcci\'on}

\begin{exercise}

(1a)\\
Respuesta a realizar.
%h (here), le decimos que ponga la imagen m\'as o menos aqu\'i
%t (top), preferiblemente en la parte superior de la p\'agina
%b (bottom), preferiblemente en la parte inferior de la p\'agina
%p (page), que junte los objetos flotantes en una p\'agina
%! que ignore sus reglas internas de posicionamiento
%H que ponga la imagen justo aqu\'i, similar a h!
\begin{figure}[h]
\centering
\begin{tikzpicture}
    \begin{axis}
    \addplot3[patch,patch refines=3,
		shader=faceted interp,
		patch type=biquadratic] 
    table[z expr=3*x^2 + 2*y]
    {
        x  y
        -2 -2
        2  -2
        2  2
        -2 2
        0  -2
        2  0
        0  2
        -2 0
        0  0
    };
    \end{axis}
\end{tikzpicture}
\caption{Gr\'afico de $f(x,y) = 3x^2 + 2y$}
\label{fig:exercice_01_a_01}
\end{figure}

\end{exercise}