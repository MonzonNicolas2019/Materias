%h (here), le decimos que ponga la imagen m\'as o menos aqu\'i
%t (top), preferiblemente en la parte superior de la p\'agina
%b (bottom), preferiblemente en la parte inferior de la p\'agina
%p (page), que junte los objetos flotantes en una p\'agina
%! que ignore sus reglas internas de posicionamiento
%H que ponga la imagen justo aqu\'i, similar a h!
\begin{figure}[h]
\centering
\begin{tikzpicture}
    \begin{axis}[axis lines=middle,
                xlabel=$x$,
                ylabel=$y$,
                enlargelimits,
                ytick=\empty,
                xtick={-1,1},
                xticklabels={$-1$,$1$}]
        \addplot[name path=H,black,domain={-1.5:1.5}] {1 + x^2} node[pos=0.8, below]{$y = 1 + x^2$};

        \addplot[name path=G,black,domain={-1.5:1.5}] {2*x^2}node[pos=0.8, above]{$y = 2x^2$};

        \addplot[pattern=north west lines, pattern color=brown!50]fill between[of=H and G, soft clip={domain=-1:1}]
        ;
        \node[coordinate,pin=60:{$\mathcal{R}$}] at (axis cs:0,0.5){};

    \end{axis}
\end{tikzpicture}
\caption{Figura de ejemplo}
\label{fig:chaper01_02_region1_example_a}
\end{figure}