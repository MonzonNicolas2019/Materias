%h (here), le decimos que ponga la imagen m\'as o menos aqu\'i
%t (top), preferiblemente en la parte superior de la p\'agina
%b (bottom), preferiblemente en la parte inferior de la p\'agina
%p (page), que junte los objetos flotantes en una p\'agina
%! que ignore sus reglas internas de posicionamiento
%H que ponga la imagen justo aqu\'i, similar a h!
\begin{figure}[ht]
\centering
\begin{tikzpicture}
    \begin{axis}[axis lines=middle,
                xlabel=$x$,
                ylabel=$y$,
                enlargelimits,
                ytick=\empty,
                xtick={-1.479,3.62193},
                xticklabels={$x_1$,$x_2$}]
        \addplot[name path=H,black,domain={-4:4}] {-(1/6)*x^2+x+2.5} node[pos=1, below]{$h$};

        \addplot[name path=G,black,domain={-4:4}] {0.3*x^2}node[pos=1, above]{$g$};

        \addplot[pattern=north west lines, pattern color=brown!50]fill between[of=H and G, soft clip={domain=-1.479:3.62193}]
        ;
        \node[coordinate,pin=60:{$A$}] at (axis cs:1.1,1.6){};

    \end{axis}
\end{tikzpicture}
\caption{Regi\'on Tipo 2 ejemplo II}
\label{fig:chaper01_02_region2b}
\end{figure}