\section{Funci\'on escal\'on (Heaviside)}

Usamos la notaci\'on $u$ por "unitario", por ser "un" escal\'on, o $h$ por \emph{Heavside}.

%Acá va la gráfica de Heaviside, y una muestra de como queda simplificada (con la línea pintada en lugar de punteada, como si no fuera una función).

$$
u(t) = \lbrace{ \begin{matrix}
0 \textrm{ si } t \leq 0 \\
1 \textrm{ si } t > 0
\end{matrix} }
$$

%Colocar los gráficos de ejemplo de las siguientes funciones:

$$f(t) = t^2u(t)$$
$$g(t) = t^2u(t+1)$$
$$h(t) = t^2u(1-t)$$
$$z(t) = t^2(u(t) - u(t-1))$$ %Es un peldaño
%Graficar la formula general u(t-a) - u(t- b) que es una onda cuadrada, que solo grafica de a hasta b.
$$w(t) = (t-1)^2 u(t-1)$$ 

\section{Transformada de Laplace}

\begin{definition}
(Transformada de Laplace) Sea $f: \mathcal{D}_f \subset \mathbb{R} \rightarrow \mathbb{R}$, se llama Transformada de Laplace (TL) de la funci\'on $f$ y se escribe $\mathcal{L}(f)(p) = \int_{0}^{\infty}e^{-pt}f(t)dt$ (si existe).
\end{definition}

Esto lo que nos dice que es que tenemos una funci\'on $f$, con una variable (por ejemplo $t$), que al ser evaluada por $\mathcal{L}$ tenemos una nueva funci\'on (llamada $F$) que tiene una nueva variable $p$.

%Colocar un ejemplo de una función, con una flecha que muestre que tiende  aun nuevo grafico de F.

Si la integral existe, entonces $f$ es $\mathcal{L}$-transformable (en caso contrario no tiene TL).
\emph{Observaci\'on}: la "cola izquierda" (si la tiene) de $f$ no se tiene en cuenta, as\'i que tenemos
$$
\int_{0}^{\infty}e^{-pt}f(t)u(t)dt
$$

Veamos un ejemplo de TL.

%Poner encima del singo igual "def"
$$
U(p) = \int_0^{\infty}e^{-pt}u(t)dt
$$
Como $u(t)$ vale 1 donde estoy integrando,
$$
= \int_0^{\infty}e^{-pt}dt = (-\frac{1}{p}e^{-pt})\|^{\infty}_0
$$
Como $p > 0$
$$
= (-\frac{1}{p}\left[0 - 1\right] = \frac{1}{p})
$$

Tenemos que pensar a $p$ como una constante que va a seguir viva luego de la integraci\'on y a $t$ como nuestra variable. La TL de $U(t)$ se define para $p>0$, y esto va a aparecer en la tabla de Laplace.

\emph{Observaci\'on} Una condici\'on suficiente para que existe la TL, es que la funci\'on $f$ sea continua por partes (CPP), y de orden $\alpha$-exponencial\footnote{Esto significa que si tiene discontinuidades, los saltos son finitos (no hay un salto infinito) y que se la puede acotar a partir de un cierto $t$ por una exponencial del tipo $e^{\alpha t}$, en tal caso, por lo menos se puede asegurar que la TL existe para $p > \alpha$}.

%%Hacer grafico explicativo con una función f(t) con una exponencial por encima y poner lo siguiente
$\|f(t)\| \leq M e^{\alpha t}$ ($M$ es un valor constante) %Verificar $\|f(t)\| \leq M e^{\alpha t}$%

%%Hacer otro grafico que muestre que para la exponencial anterior, F(p) tiene una absisa de convergencia, es decir una linea vertical en \alpha que muestra que se grafica desde allí hacia la derecha.

%%Colocar apéndice con tablas de Laplace

Propiedades:

\begin{enumerate}

\item (Verificaci\'on de la transformada) Suponiendo que $f(t)u(t) \longrightarrow F(p)$ %colocar encima de la flecha TL
$tf(t)u(t) \longrightarrow ?$ %colocar encima de la flecha TL, signo de pregunta en rojo
Veamos:
$$
F(p)= \int_0^{\infty}e^{-pt}f(t)dt
$$
Derivamos esta funci\'on (en funci\'on de $p$)
$$
\frac{d}{dp}F(p) = \frac{d}{dp}\int_0^{\infty}e^{-pt}f(t)dt = \int_0^{\infty}\frac{d}{dp}(e^{-pt}f(t))dt
$$
Es decir, pensamos que nuestra integral es transparente a nuestra derivaci\'on, lo cual se debe a propiedades de continuidad que no se van a tocar. Suponemos que esto es verdad por el momento,
$$
= \int_0^{\infty}\frac{d}{dp}((-t) e^{-pt}f(t))dt = -F^{\prime}(p)
$$
Entonces $tf(t)u(t) \longrightarrow -F^{\prime}(p)$. %colocar encima de la flecha TL
Tambi\'en se tiene, en general %%Poner todo esto en una tabla como en mi tesis
$t^n f(t)u(t) \longrightarrow (-1)^n F^{(n)}(p)$.
Por ejemplo: $tu(t) \longrightarrow -(\frac{1}{p}^{\prime}) = -(-\frac{1}{p^2}) = p^{-2}$.

\item $f(t)u(t) \longrightarrow F(p)$, entonces $f^{\prime}(t) = ?$
$$
\mathcal{L}(f^{\prime}(p) = \int_0^{\infty}e^{-pt}f^{\prime}(t)dt
$$%Colocar "def" encima de la flecha
Integro por partes con $u = e^{-pt}$, $du = -p e^{-pt}dt$, $dv = f^{\prime}(t)dt$, $v = f(t)$
$$
=f(t)e^{-pt}\|^{\infty}_0 + \int_0^{\infty}pe^{-pt}f(t)dt = pF(p) - f(0)
$$
Entonces $f^{\prime}(t) = pF(p) - f(0)$. De ac\'a, se tiene que $f^{\prime\prime}(t) \longrightarrow p^2F(p)-pf(0)-f^{\prime}(0)$ %colocar encima de la flecha un TL
Es decir, multiplicamos por $p$ lo que ya teniamos y le restamos lo que acabamos de derivar, en $0$. Para $f^{\prime\prime\prime}(t) \longrightarrow p^3F(p)-p^2f(0) - pf^{\prime}(0) -f^{\prime\prime}(0)$.

\item $f(t)u(t) \longrightarrow F(p)$, entonces $e^{at}f(t)u(t) \longrightarrow ?$ %Colocar TL encima de la flecha.
Calculamos:
$$
\mathcal{L}(e^{at}f(t)u(t))(p) =\int_0^{\infty}e^{-pt}e^{at}f(t)dt = \int_0^{\infty}e^{-(p-a)t}f(t)dt = F(p-a)
$$
$u(t)$ vale $1$ en el \'area de integraci\'on. Entonces $e^{at}f(t)u(t) \longrightarrow F(p-a)$.

Con la propiedad anterior, podemos por definici\'on de $\cosh$ y $\sinh$:
$$
\cosh{at} = \frac{1}{2}(e^{at} + e^{-at}) \longrightarrow \frac{1}{2}(\frac{1}{p-a} + \frac{1}{p+a}) = \frac{p}{p^2-a^2},p>\|a\|
$$ (esto \'ultima dado que por la primera fracci\'on, $p > a$ y por la segunda $p > -a$). %Colocar TL encima de la flecha
$$
\sinh{at} = \frac{1}{2}(e^{at} - e^{-at}) \longrightarrow \frac{1}{2}(\frac{1}{p-a} - \frac{1}{p+a}) = \frac{a}{p^2-a^2},p>\|a\|
$$ (esto \'ultima dado que por la primera fracci\'on, $p > a$ y por la segunda $p > -a$). %Colocar TL encima de la flecha

Podemos combinar propiedades:

$$
t e^{at} u(t) \longrightarrow -\frac{1}{p-a} = \frac{1}{(p-a)^2}
$$ %Colocar TL encima de la flecha

\item (Desplazamiento de la original)

Aca poner lo de las fotos del celu (Lo que escribi\o el profesor en hojas verdes de UADE)

\end{enumerate}

\section{Ejercicios}

\begin{enumerate}

\item Ejercicio 1
\item Ejercicio 2
\item Ejercicio 3 ...

\end{enumerate}

\section{Claves de correcci\'on}

\begin{exercise}
(2a)
$$x = e^{t+1}$$
$$x^{\prime} = e ^{t + 1} = x$$
$$x = x^{\prime}$$
Entonces: $x^{\prime} - x = 0$ EDO.

\end{exercise}

\begin{exercise}
(2c)
$$x = A \sin{t}$$
$$x^{\prime} = A \cos{t}$$
$$x^{\prime \prime} = - A \sin{t}$$
Entonces: $x + x = x^{\prime \prime} = 0$.

\end{exercise}

\begin{exercise}
(2f) Derivamos dos veces para conseguir dos consntantes. La ecuaci\'on diferencial de esa familia ($x = x(t)$).  Derivamos dos veces la familia y eliminamos $c_1$ y $c_2$ (respecto de $t$).

$$
\frac{1}{x}x^{\prime} = 2c_1t
$$
y entonces
$$
\frac{x^{\prime\prime}x - x^{\prime^2}}{x^2} = 2c_1
$$

La ED es

$$
\frac{x^{\prime}}{x} = \frac{x^{\prime\prime}x - x^{\prime^2}}{x^2}t
$$

Como no se pide nada m\'as, no continuamos.

\end{exercise}

\begin{exercise}
(3e)

Como nos piden verificar, solo derivamos y reemplazamos. Con $y = y(x)$ en la EDO $1^3 + x = x + 1 = y$, por lo tanto, cumple.
$y = x + 1$ es soluci\'on particular (no hay par\'ametros libres).

\end{exercise}

\begin{exercise}
(4) Incompleto.

Consideramos $y^{\prime} = \frac{dy}{dx}$

\end{exercise}

\begin{exercise}

(6h)

$$
x^{-2}x^1 + \frac{1}{x} = t
$$
Para la sustituci\'on
$$
y^{\prime}(t) = -\frac{1}{x^2}x^{\prime}
$$

$$
-y^{\prime}(t) + y(t) = t
$$

$$
y^{\prime}(t) - y(t) = -t
$$

Factor integrante $\mu(t)=e^{-\int dt}  =e^{-t} \neq 0$, $\forall t \in  \mathbb{R}$. Multiplicando por $e^{-t}$:

$$
e^{-t}y^{\prime}(t) - e^{-t}y(t) = -te^{-t}
$$

$$
(e^{-t}y(t))^{\prime} = -te^{-t}
$$

Pero si esto se cumple, entonces tenemos una primitiva tal que

$$
e^{-t}y(t) = -\int te^{-t}dt = t e^{-t} + e^{-t} + c
$$

Finalmente, $y: \mathbb{R} \rightarrow \mathbb{R}$, $y(t) = t + 1 + ce^t$ de modo que la CI dice que $y(1) = 1$, entonces debe pasar que $1 + 1 + ce = 1$ entonces $c = -\frac{1}{e}$ de modo que $y: \mathbb{R} \rightarrow \mathbb{R}$, $y(t) = t + 1 - \frac{1}{e}e^t$. Pero lo que nos piden es $x$, entonces:
$I \subset \mathbb{R} \rightarrow \mathbb{R}$, $x(t) = \frac{1}{1 + t - \frac{1}{e}e^t}$.

\end{exercise}

\begin{exercise}
(7h) Considerando $y^{\prime} + p(x)y = q(x)$
$$y \longrightarrow 1$$
$$x \longrightarrow \infty$$
Tenemos
$$x^{2}y^{\prime} + xy^{\prime} = y - 1$$
$$(x^2 + x)y^{\prime} - y = -1$$
$$y^{\prime} - \frac{1}{x^2 + x}y = - \frac{1}{x^2 + x}$$
Tomamos las funciones
$$g(x) = - \frac{1}{x^2 + x}$$
$$f(x) = - \frac{1}{x^2 + x}$$
Por \emph{factor integrante},
$$\mu(x) = e ^{\int{- \frac{1}{x^2 + x}dx}}$$

C\'alculo auxiliar.

$$
{\int{- \frac{1}{x^2 + x}dx}} = 
{\int{- \frac{1}{x(x + 1)}dx}} =
{\int{(\frac{A}{x} + \frac{B}{x+1})dx}} =
{\int{(\frac{A(x+1) + BX}{x(x+1)}})dx}
$$
$$
A(x+1) + Bx = 1 
$$
Uso $x = 0$, entonces $A = 1$. Si $x = -1$, entonces $B = -1$
$$
={\int{(\frac{1}{x} + \frac{1}{x+1})dx}}
=\ln{x}-\ln{x+1}=ln{\frac{x}{x+1}}
$$

Fin del c\'alculo auxiliar.

Entonces
$$
\mu(x)=e^{-\ln{\frac{x}{x+1}}} = \frac{x}{x+1}^{-1} = \frac{x+1}{x} = 1 + \frac{1}{x}
$$
$$
y = \frac{\int{-\frac{x+1}{x} \frac{1}{x^2 + x}}}{\frac{x+1}{x}}
$$

Partiendo de $(x^2 + x)y^{\prime} - y = 1$ entonces $(x^2 + x)\frac{dy}{dx} = y - 1$, entonces $\frac{dy}{y-1} = \frac{dx}{x^2 + x}$ y, entonces, $\ln{\|y-1\|} = \ln{\|\frac{x}{x+1}\|} + C_1$. Darnos cuenta de que son variables separadas nos da una ventaja m\'as grande. Una forma m\'as r\'apida es ver que si $x\longrightarrow \infty$, entonces dado que $y - 1 = k\frac{x}{x+1}$, entonces, si $y = 1 + k\frac{x}{x+1}$ nos queda $y = 1$. Y es f\'acil verificar entonces que $y^{\prime} = 0$ y cumple la condici\'on inicial.

\end{exercise}

\begin{exercise}

(10b) Despejo de (1) la variable $y = 2x - \dot{x} -3$, reemplazando en (2) es $2\dot{x} - \ddot{x} = -x + 4x - 2\dot{x} - 6$ entonces queda $\ddot{x} - 4\dot{x} + 3x - 6$ entonces una ecuaci\'on diferencial completa de 2° orden con coeficientes constantes 

(a) la homog\'enea es $\ddot{x} - 4\dot{x} + 3x = 0$, ecuaci\'on caracter\'istica $\lambda ^2  4\lambda + 3 = 0$, $\sigma = \lbrace \lambda_1 = 1, \lambda_2 = 3$. Soluci\'on de la homog\'enea $x_h(t) = c_1e^t + c_2e^{3t}$.

(b) Una soluci\'on particular $x_p(t) = k_1$ con $k_1$ tal que $0 - 4(0) + 3k_1 = 6$, $k_1 = 2$. La soluci\'on particular es $x_p(t)=2$.

(c) La soluci\'on general es $x(t) = c_1e^t + c_2e^{3t} + 2$.

Ahora $y = 2c_1e^t + 2c_2e^{3t} + 4 - (c_1e^t + 3c_2e^{3t}) - 3 = c_1 e^t - c_2e^{3t} + 1$ (Aca poner una llave por debajo que muestre que $2x = y = 2c_1e^t + 2c_2e^{3t} + 4$  y $\dot{x} = c_1e^t + 3c_2e^{3t}$).

La soluci\'on general

%\usepackage{amsmath}
\begin{equation}
\begin{cases}
	x(t) = c_1e^t + c_2e^{3t} + 2 \\
	y(t) = c_1e^t - c_2e^{3t} + 1	
\end{cases}
\end{equation}

%Flecha que diga CI por Condiciones Iniciales

\begin{equation}
\begin{cases}
	c_1 + c_2 + 2 = 2 \\
	c_1 - c_2 + 1 = 1
\end{cases}
\end{equation}

%Flecha

$$
c_1 = c_2 = 0
$$

Finalmente, la soluci\'on del problema es

\begin{equation}
\begin{cases}
	x: \mathbb{R} \rightarrow \mathbb{R}, x(t) = 2 \\
	y: \mathbb{R} \rightarrow \mathbb{R}, y(t) = 1
\end{cases}
\end{equation}

\end{exercise}

\begin{exercise}

(Problema de parcial, con enunciado) Dada la familia $\Phi = \lbrace (x,y) \in \mathbb{R}^2 : kx^2 + 1 - y = 0 \rbrace$ hallar la familia de trayectorias ortogonales $\Psi$ y la ecuaci\'on de la \'unica curva $C \in \Phi$ que pasa por $P_0 = (\sqrt{2}, 0)$ y graficarla.

\emph{Respuesta}

La familia $\Phi$ es una familia de par\'abolas de ecuaci\'on $y = 1 + kx^2$.

(1) Derivo la familia original y elimino $k$.
\begin{equation}
\begin{cases}
	y = 1 + kx^2 \\
	y^{\prime} = 2kx
\end{cases}
\end{equation}

Eliminando $k$ queda $\frac{y - 1}{y^{\prime}} = \frac{x}{2}$.

(2) Transformamos a la ecuaci\'on diferencial de la familia $\Psi$.

$$
-(y-1)y^{\prime} = \frac{x}{2}
$$

(3) Resolvemos la ecuaci\'on anterior

$$
-(y-1)\frac{dy}{dx} = \frac{x}{2}
$$

O bien

$$
\frac{x}{2}dx + (y-1) dy = 0
$$

$$
d\left[ \frac{x^2}{4} + \frac{1}{2}(y-1)^2\right] = 0
$$

Entonces

$$
\frac{x^2}{4} + \frac{(y-1)^2}{2} = 0
$$

Es decir, la familia de las curvas ortogonales son todas elipses. %Colocar dibujo de varias elipses y varias curvas para que se vea que se cortan ortogonalmente

¿Cu\'anto vale $c$ para que $\frac{x^2}{4} + \frac{(y-1)^2}{2} = c$ pase por $(\sqrt{2}, 0)$? Reemplazamos y nos queda $c = 1$.

Luego la elipse es

$$
\frac{x^2}{4} + \frac{(y-1)^2}{2} = 1
$$

Al final del pizarr\'on hab\'ia una nota que dec\'ia "Cambiar $y^{\prime}$ por $\frac{1}{y^{\prime}}$".

\end{exercise}

\begin{exercise}

(Ejercicio de parcial)

\begin{equation}
\begin{cases}
	\dot{x} = 2x - y + 1 & (1) \\
	\dot{y} = 2x - y + 2 & (2)
\end{cases}, x(0) = y(0) = 2
\end{equation}

\emph{Respuesta}

De la ecuaci\'on (1) despejo $y = 2x - \dot{x} + 1$ y reemplazo en (2)
$$
2\dot{x} - \ddot{x} = 2x - 2x + \dot{x} -1 + 2
$$
Reescribimos esto como una ED completa de 2° orden: $\ddot{x} - \dot{x} = -1$ (ED completa).

(a) La homog\'enea $\ddot{x} - \dot{x} = 0$ con ecuaci\'on caracter\'istica $\lambda^2 - \lambda = 0$ con espectro $\gamma = \lbrace \lambda_1 = 0, \lambda_2 = -1 \rbrace$ soluci\'on $x_h(t) = c_1 + c_2 e^t$.

(b) Ac\'a tenemos que tener cuidado con el grado del polinomio, hay que fijarse en el espectro. Soluci\'on particular $x_p(t) = k_1t$. $$0 - k_1 = -1 \rightarrow k_1 = 1$$ Luego $x_p(t) = t$.

(c) Soluci\'on general $x(t) = c_1 + c_2e^t + t$.

Ahora $y(t) = 2c_1 + 2c_2 e^t + 2t - (c_2 e^t + 1) + 1 = 2c_1 +c_2 e^t + 2t$. Luego, la soluci\'on es

\begin{equation}
\begin{cases}
	x(t) = c_1 + c_2 e^t + t \\
	y(t) = 2c_1 + c_2 e^t + 2t
\end{cases}
\end{equation}

Aplicamos CI

\begin{equation}
\begin{cases}
	c_1 + c_2 = 2 \\
	2c_1 + c_2 = 2
\end{cases}
\end{equation}

Entonces $c_1 = 0$ y $c_2 = 2$. La \'unica soluci\'on es

\begin{equation}
\begin{cases}
	x(t) = 2e^t + t \\
	y(t) = 2e^t + 2t
\end{cases}
\end{equation}

\end{exercise}

\begin{exercise}

(Ejercicio de parcial)

$\ddot{x} - 4\dot{x} + 3x = 5 - 8t + 3t^2$, $x(0) = 1$, $\dot{x}(0) = 0$.

(a) La homog\'enea $\ddot{x} - 4\dot{x} + 3x = 0$ con ecuaci\'on caracter\'istica $\lambda^2 - 4\lambda + 3 = 0$ con espectro $\gamma = \lbrace \lambda_1 = 1, \lambda_2 = 3 \rbrace$.
$$
x_h(t) = c_1e^t + c_2e^{3t}
$$

(b) Soluci\'on particular $x_p(t) = k_1 + k_2t + k_3 t^3$ con $k_1$, $k_2$, $k_3$ tales que $2k_3 = \ddot{x}$, $\dot{x} = k_2 - 2k_3$, entonces
$$
2k_3 - 4k_2 - 8k_3 t + 3k_1 + 3k_2 t + 3k_3 t^2 = 5 - 8t + 3t^2
$$

\begin{equation}
\begin{cases}
	2k_3 - 4k_2 + 3k_1 &= 5 \\
	-8k_3 - 4k_2 &= -8 \\
	3k_3 &= 3
\end{cases}
\end{equation}

Entonces $k_3 = 1$, $k_2 = 0$ y $k_1 = 1$.

La soluci\'on general es $x: \mathbb{R} \rightarrow \mathbb{R}$ tal que $x(t) = c_1e^t + c_2e^{3t} + 1 + t^2$, ahora, imponemos las CI. Nos queda que $c_1 = c_2 = 0$. Finalmente, la \'unica soluci\'on es $x(t) = t^2 + 1$.
 
\end{exercise}

\begin{exercise}

(Ejercicio de parcial que quiz\'as est\'a en la gu\'ia)

Familia de curvas.

$x^2 + y^2 = 2cx$, $P_0 = (0,2)$.

\emph{Respuesta}

Completamos cuadrados.

$$
x^2 -2cx + y^2 = 0
$$

$$
(x-c)^2 + y^2 = c^2
$$

Cuando graficamos nos damos cuenta que son c\'irculos cnetrados en 0, y desplazados lo mismo que sus radios. %Colocar gr\'afico de varios c\'irculos.

(1) Derivamos y eliminamos $c$.

\begin{equation}
\begin{cases}
	x^2 + y^2 = 2cx \\
	2x + 2yy^{\prime} = 2c
\end{cases}
\end{equation}

Eliminando $c$

$$
x^2 + y^2 = (2x + 2yy^{\prime})x
$$

O bien $-x^2 + y^2 = 2xyy^{\prime}$

(2) Ahora pasamos a la ED de las trayectorias ortogonales.

$$
-x^2 + y^2 = \frac{2xy}{y^{\prime}}
$$

O bien 

$$
(-x^2 + y^2)y^{\prime} = -2xy
$$

(3) Resolvemos (2) $2xydx + (-x^2 + y^2)dy = 0$ es exacta (NO). ¿Es reducible a exacta? SI.
Nos queda algo como sigue:
$$
x^2 + (y - k)^2 = k^2
$$
Con $k = 1$, $x^2 + (y - 1)^2 = 1$.

\end{exercise}