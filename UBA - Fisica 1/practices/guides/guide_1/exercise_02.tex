\begin{enumerate}

\item

	Velocidad en funci\'on del tiempo:

	$$ v = \dv{x}{t} = \dv{\sqrt{x_0^2+2kt}}{t} $$
	$$ v = \frac{1}{2}\frac{2k}{\sqrt{x_0^2+2kt} }$$
	$$ v(t) = \frac{k}{\sqrt{x_0^2+2kt}} $$
	
	Aceleraci\'on en funci\'on del tiempo:
	
	$$ a = \dv{v}{t} = \dv{\frac{k}{\sqrt{x_0^2+2kt}}}{t} $$
	$$ a = \frac{0 - \frac{k^2}{\sqrt{x_0^2+2kt}}}{x_0^2+2kt} $$
	$$ a(t) = -\frac{k^2}{(x_0^2+2kt)^\frac{3}{2}} $$
	
\item
	
	En $t = 0$, dado que $x_0 > 0$, es v\'alido lo siguiente:
	$$ x(t = 0) = \sqrt{x_0^2} = \left| x_0 \right| = x_0 $$
	
	Velocidad:
	
	$$ v(x) = \frac{k}{x}$$
	
	Aceleraci\'on:
	
	$$ a(x) = -\frac{k^2}{x^3}$$
	
	Los gr\'aficos aproximados deben de ser tendiendo a cero mientras mas grande es el valor de $x$ y tendiendo a $\inf$ cuando el valor este m\'as pr\'oximo del $0$. La velocidad siempre ser\'a positiva y la aceleraci\'on negativa. 

\end{enumerate}