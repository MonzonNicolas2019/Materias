\begin{enumerate}

\item

	Por definici\'on, $\frac{dx}{dt} = v$ y $\frac{dx^2}{d^2t} = \frac{dv}{dt} = a$. Entonces:

	Velocidad:

	$$ \frac{dx}{dt} = \frac{d(-kt^3 + bt^2)}{dt} $$
	$$ v = -3kt^2 + 2bt $$

	Gr\'afico de la velocidad en funci\'on del tiempo, con $k=1$

	%h (here), le decimos que ponga la imagen m\'as o menos aqu\'i
%t (top), preferiblemente en la parte superior de la p\'agina
%b (bottom), preferiblemente en la parte inferior de la p\'agina
%p (page), que junte los objetos flotantes en una p\'agina
%! que ignore sus reglas internas de posicionamiento
%H que ponga la imagen justo aqu\'i, similar a h!
\begin{figure}[h]
\centering
\begin{tikzpicture}
    \begin{axis} [xlabel = $t$, ylabel = $b$]
    \addplot3[patch,patch refines=3,
		shader=faceted interp,
		patch type=biquadratic] 
    table[z expr=-3*x^2 + 2*y*x]
    {
        t  b
        -2 -2
        2  -2
        2  2
        -2 2
        0  -2
        2  0
        0  2
        -2 0
        0  0
    };
    \end{axis}
\end{tikzpicture}
\caption{Gr\'afico de $f(b,t) = -3t^2 + 2bt$}
\label{fig:exercice_01_a_01}
\end{figure}

	Aceleraci\'on:

	$$ \frac{dv}{dt} = \frac{d(-3kt^2 + 2bt)}{dt} $$
	$$ a = -6kt + 2b $$

	Gr\'afico de la aceleraci\'on en funci\'on del tiempo, con $k=1$

	%h (here), le decimos que ponga la imagen m\'as o menos aqu\'i
%t (top), preferiblemente en la parte superior de la p\'agina
%b (bottom), preferiblemente en la parte inferior de la p\'agina
%p (page), que junte los objetos flotantes en una p\'agina
%! que ignore sus reglas internas de posicionamiento
%H que ponga la imagen justo aqu\'i, similar a h!
\begin{figure}[h]
\centering
\begin{tikzpicture}
    \begin{axis} [xlabel = $t$, ylabel = $b$]
    \addplot3[patch,patch refines=3,
		shader=faceted interp,
		patch type=biquadratic] 
    table[z expr=-6*x + 2*y]
    {
        t  b
        -2 -2
        2  -2
        2  2
        -2 2
        0  -2
        2  0
        0  2
        -2 0
        0  0
    };
    \end{axis}
\end{tikzpicture}
\caption{Gr\'afico de $f(b,t) = -6t + 2b$}
\label{fig:exercice_01_a_01}
\end{figure}

\item

	Si $v = 0$ entonces
	
	$$ -3kt^2 + 2bt = 0$$
	
	Aplicando la \emph{f\'ormula de Bhaskara} (de ahora en m\'as \emph{resolvente}), nos queda
	
	$$ t_1, t_2 = \frac{-2b \pm 2b}{-6k} $$
	
	Es decir,
	
	$$ t_1 = 0$$
	$$ t_2 = \frac{2}{3} \frac{b}{k}$$
	
	Ahora podemos encontrar la respectiva posici\'on:
	
	$$ x(t_1) = 0 $$
	$$ x(t_2) = -k \frac{2b}{3k}^3 + b\frac{2b}{3k}^2 = \frac{4b^3}{27k^2}$$
	
\item

	Con $k = 0$, la aceleraci\'on sera siempre 0 o positiva. \\
	Con $b = 0$, la aceleraci\'on sera siempre 0 o negativa. \\
	Con $t < 0$ la aceleraci\'on sera siempre 0 o positiva. \\
	Con $t = 0$ la aceleraci\'on es 0.

\end{enumerate}